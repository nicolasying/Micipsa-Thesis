
\begin{table}

    \centering
    
    \begin{tabularx}{\textwidth}{@{} A{4cm} *{5}{X} @{}}
    \toprule
    \tabhead{Reference} & \tabhead{Frontal Lobe} & \tabhead{Temporal Lobe} & \tabhead{Parietal Lobe} & \tabhead{Occipital Lobe} & \tabhead{Limbic Lobe}\\
    \midrule
    \textcite{tsukiuraDissociableRolesBilateral2006} & IFG & CA, STG & AG & GF & PCC \\
    \textcite{pobricCategorySpecificCategoryGeneralSemantic2010} &  & ATL & IPL \\
    \textcite{turkenNeuralArchitectureLanguage2011} & IFGorb, MFG & left pMTG, aSTG, pSTS, BA39 & \\
    \textcite{huthContinuousSemanticSpace2012} & Frontal eye field, frontal operculum, supplementary eye fields \dots & pITS, pSTS, STG, Heschel \dots & intraparietal sulcus & V1--V4, VO, V7, V3A/B & middle cingulate gyrus/sulcus \\
    \textcite{huthNaturalSpeechReveals2016} & SPFC, IPFC & LTC, VTC & LPC, MPC & \\

    \bottomrule \\
    \end{tabularx}
    \caption{Involvement of Cerebral Areas in Semantic Tasks}
    \label{tab:distributedareastudysynthesis}
    \end{table}