% Introduction 


% The \emph{distributed-plus-hub} view is backed by the necessity of high-order generalization and abstraction of concepts despite their particular perceptual attributes.  \cite{damasioNeuralBasisLexical1996, damasioNeuralSystemsWord2004} identified regions beyond Wernicke's and Broca's area in the retrieval of nominal concepts using lesion studies and PET [TO Read Dasiamo 2004]. Those regions in question, particularly located at the both hemisphere in temporal lobe, are implicated in the recognition of faces, animals and tools (right hemisphere) and the retrieval of corresponding lexicons (left hemisphere). Semantic demantia, as a result of ATL lesions, and other diseases related to cortical area damage and atrophy, provide a strong link of inferior anterior temporal lobe and specific categorizations [TO Clarify].

% A meta-analysis of 

% [Insert ROI peaks FIG]

% From morphemes, lexicons, phrases then sentences, be they procounced, written or brailled, each is linked to its simple or complex meaning and its manifestation in the world. In the current usage of languages, lexicons are often the minimal semantic unit that we employ to express ourselves, as coinage is much more rare than formulating a phrase out of words. If we adopt the semantic set theory [REF needed, Mental Space e.g.], an adjective such as "brown", a noun such as "cat", a verb such as "run", and an adverb such as "quickly" respectively represent the property of sharing the colour with chocolate, the taxonomical family of animals, the forward movement upon two or four feet "by alternately making a short jump off either foot" [wiktionary], and the property of an action that is executed in a shorter time than normality. Upon the composition of "brown" and "cat" to form "brown cat" and similarly for "run quickly", the represented concept sets get refined by applying an intersection of the associated object sets. In the meantime, such composited meanings often have their proper univerbal lexicons, such as "rush" for "run quickly", though the meaning are often not an exact match.



% [Need Precision] If we theorize that semantic memories are the neuro-representations encoded by coordinated neurons, how does the brain store millions of different concepts that we know of, and provide a stable mechanism for lexico-conceptual access and retrieval? 

% [Paragraph for whole-axe studies]


% When humans query a dictionary to get the meaning of a certain lexicon, such case could arrive that they get into a recursive loop if their vocabulary is limited so that they do not understand the words employed to describe the meaning of the original lexicon. For a computational system whose lexical knowledge cannot be assumed, a relational knowledge base skips the step of building a fundamental vocabulary, but render the relationships between all possible concepts/lexicons. Such base gives reference in an affine space, so knowledges are represented by its relations to all other concepts. WordNet-like \parencite{millerWordNetLexicalDatabase1995, millerWordNetElectronicLexical1998} knowledge bases are examples of such a system.