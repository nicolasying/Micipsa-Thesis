\chapter{Hypotheses} % Main chapter title

\label{chap:hypotheses} % For referencing the chapter elsewhere, use \ref{Chapter1} 

\section{Reconciliation of Multiple Theories Exhibiting the Twofold Character of Natural Languages}

In continuity with \cite{jakobsonFundamentalsLanguage1963}'s syntagmatic and paradigmatic axes and multiple theories on the role of cortices which are not traditionally defined as language areas, we focus on dissecting a central semantic locus, which acts as the binder, gateway or hub in different theories' nomination, apart from other peripheral semantic components. With the key question on finding the semantic information encoded by various cortical departments engaged in semantic processing, we propose a possible construction of the semantic hub based on the following arguments.

\subsection{Definition of Semantic Similarity and Semantic Hub}

\label{subsection:hypsemantichub}

[Checklist: cross-modal/amodal, hierarchical, ]
If a semantic hub exists, as argued by \cite{pattersonWhereYouKnow2007, ralphNeuralComputationalBases2017} [TODO, many other references], the semantic hub is more aligned with Jakobson's paradigmatic axis. 

Additionally, pathological evidences from \cite{pattersonWhereYouKnow2007} suggest that \emph{semantic similarity} implements a hierarchical structure: \emph{herpes simplex virus encephalitis} (HSVE) patients compared with \emph{semantic demantia} (SD) patient show an intact performance at basic semantic levels (such as dog, knife) but not at subordinate level (for example poodle or bread knife). [TODO More evidence needed]

We assume the amodality and potential hierarchy of semantic similarity is well conserved in ontological semantic networks introduced in section \ref{subsection:symbolicembedding}. Furthermore, WordNet-like networks explicit various semantic relationships, some of which are considered as the backbone structure of similarity hierarchy. We will approximate a pure similarity space, using only similarity-related semantic relations in WordNet. 

\subsection{Definition of Semantic Association}  
[TODO definition ? alternative ]
[Checklist: modal-specific, syntax-based proximity, ]
By pathological studies, the semantic hub's information structure (which we name henceforth \emph{semantic similarity}) should be able to attribute to metaphorical word-pairs (such as \emph{spyglass}/\emph{microscope}). 

\subsection{Hypothetical Structure of Twofold Characters in Statistical Distributional Representations}

\label{subsection:hyplinearsemantics}

 

We furthered the argument of the mixture of semantic association and semantic information in statistical distributional representation in section \ref{subsection:statisticalembedding} on assuming a linear additive structure of the two components. Linear structuralism is often adopted in semantic modeling [TODO ref, eg. CBOW of mikolov]. It achieves adequate performance despite its potential inexactitude. 

Under this assumption, we can obtain a semantic association representation space by subtracting similarity representations from a mixed representation space.

\section{Targeted Semantic Hub Locus}

bilatearl Temporal Lobe / vlATL [TODO detail]

Theoretical reconstruction of cerebral activity based on semantic similarity embeddings should found a better modeling performance in semantic hub regions, whereas association embedding reconstructions in other language areas located in other [TODO which \emph{other}?] regions. [TODO rephrase]