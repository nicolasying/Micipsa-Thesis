\chapter{Hypotheses} % Main chapter title
\label{chap:hypotheses} % For referencing the chapter elsewhere, use \ref{Chapter1} 

To underpin the internal semantic representation structure of the argued semantic hub \parencite{pattersonWhereYouKnow2007}, thus correspondingly that of other non-hub components of the semantic processing neural network, we relate \citeauthor{desaussureCoursLinguistiqueGenerale1969}, \citeauthor{jakobsonFundamentalsLanguage1963}'s twofold structuralism with neuro-psycholinguistic theories on semantic processing, notably the \emph{Hub-and-Spoke} and the \emph{Controlled-Semantic-Cognition} theories~\parencite{lambon-ralphNeuralComputationalBases2017}. Two new terms \similarity and \association are employed in this project to bridge two fields, representing two parallel (not necessarily separate) systems that handle respectively metaphorical and associational access, retrieval and processing of word-meanings in linguistic tasks.

\section{Reconciliation of Multiple Theories Exhibiting the Twofold Character of Natural Languages}

We focus on dissecting a central semantic locus, which acts as the binder, gateway or hub in different theories' nomination, which is activated in all types of semantic processing, apart from other peripheral semantic components, in aide of paradigmatic and syntagmatic semantic representation models applied in an fMRI encoding experiment. 

\subsection{Semantic \emph{Similarity} and Semantic Hub} 
\label{subsection:hypsemantichub}

% [TODO, convergence on amodal, quasimetaphoric]

The paradigmatic axis is associated with the semantic hub and the proposed semantic \similarity principle.
The semantic hub is the locus\slash loci where the ontological semantic information of all words is represented.  Such a particular ontology encodes the human understanding of concepts free of the dominant influence of modality-specific semantics. The plausibility of such a hub is motivated by contexture-deficients' quasimetaphoric wordings (which is a paradigmatic property) and selection-deficients' impaired object naming ability limited to associates (Section \ref{sec:IntroSyntagandParaAxies}). \textcite{pattersonWhereYouKnow2007} also summarized symptoms including concept retrieval and categorization difficulties. 

Evidences from SD studies indicate three principle factors in semantic hub organization: familiarity, typicality and specificity.~\parencite{pattersonWhereYouKnow2007} Familiarity is constructed with episodic events (thus out of scope of this project). Typicality can be encoded in the semantic ontology since untypical concepts usually require more information (e.g. \emph{whale} is conceptually very similar to other marine fishes, thus it needs to be marked as an exception in the semantic system since it is a mammal). Specificity can be modeled by a hierarchical structure, where the access to a word is an iterated traverse in a semantic tree. Hierarchical structures also allow to cope with a large lexicon inventory. Evidences support this hierarchy account. Specific (e.g. \emph{robin})~\parencite{rogersAnteriorTemporalCortex2006}, basic-level (e.g. \emph{dog}) and domain-level (e.g. \emph{animal})~\parencite{pobricCategorySpecificCategoryGeneralSemantic2010} semantic information are available from the argued semantic hub locus. 

For computational implementation of \similarity, we assume the amodality and potential hierarchy of semantic \similarity is well conserved in ontological semantic networks introduced in Section \ref{subsection:symbolicembedding}. WordNet-like networks hand-code its semantic units, introducing thus a familiarity bias. Furthermore, they explicit various semantic relationships: hypernymy and hyponymy are considered as the backbone structure of similarity hierarchy, synonymy (formed by \emph{synsets}) pushes similar words closer\dots

The internal organization of word-meanings in the semantic hub are henceforth named \emph{similarity}. It has a more global view towards all word meanings, whereas paradigmatic relations are a local similarity manifestation. 

\subsection{Semantic \emph{Association}}

The syntagmatic axis is related with semantic control, episodic-event based (phrasal usage or personal experience) semantic information. As an umbrella term, we consider \association as complementary to \similarity in semantic memory. It includes thus syntagmatic relations, modality-specific proximities and episodic associations. Extended syntagmatic relations include collocations (\emph{pencil} and \emph{write}), meronymy\slash holonymy (\emph{ceil} and \emph{house}), entailment\slash causality (\emph{sunset} and \emph{milky-way}). Modality-specific proximities include spatial proximities (\emph{bridge} and \emph{river}), visual similarity (Paris metro logo and McDonald's), rhymes (\emph{rhyme} and \emph{lime})\dots

The domain- and feature-specific theories on distributed semantic processing is also backed by \emph{association}. Domain-specific information such as \emph{tools} recruit sensory-motor functions, \emph{music instruments} require auditory functions, and \emph{human faces} calls for affectual and social departments. 

\emph{Association} is vast. In this project we model the \association axis by exploiting all available information from statistical distributed representations (SDR), in which syntagmatic relations, common episodic associations (manifested in the SDR source corpus) are present.

\section{Approximative Structure of Twofold Characters in Statistical Distributional Representations}

\label{subsection:hyplinearsemantics}
As SDR is a mixture of \similarity and \association information, we approximate this mixture by a linear additive structure of two components. Despite its simplicity, linear structuralism is often adopted in semantic modeling (e.g. Continuous-Bag-of-Words which is proposed in \textcite{mikolovEfficientEstimationWord2013}), and it achieves adequate performance. 

With this approximation, we can obtain a semantic \association representation space by subtracting \similarity representations from a mixed representation space. Once the \similarity component removed, the embedding should rank associates above the residual synonymies.

\section{Targeted Semantic Hub Locus}

With the key question on finding the semantic information encoded by various cortical departments engaged in semantic processing, we propose to test the accuracy of semantic models by reconstructing the semantic hub\slash non-hub contrast maps with proposed embeddings for each component. If the so argued semantic hub exists, and our hypotheses on semantic hub's internal structure are accurate, the theoretical reconstruction of cerebral activity with fMRI encoding based on semantic \similarity embeddings should better model semantic hub region activities, whereas \association embedding reconstructions in other language areas. The embedding model performance map should somehow align with the hub and non-hub component spatial pattern.

As classical view holds that the temporal cortex hosts semantic memory, more precise semantic hub loci are proposed by different theories. \textcite{priceMetaanalysesObjectNaming2005, pattersonWhereYouKnow2007, binderMappingAnteriorTemporal2011} argue for an bilateral anterotemporal (aTL) loci, whereas \textcite{lambon-ralphNeuralComputationalBases2017} refined the search to ventrolateral aTL. As only at a large-scale is the spatial placement of anatomical convergence zone predictable across individuals~\parencite{damasioNeuralSystemsWord2004}, we ground our targeted localization precision to aTL.