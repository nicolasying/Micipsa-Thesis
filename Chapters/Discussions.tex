\chapter{Discussion} % Main chapter title

\label{chap:discussions} 

%----------------------------------------------------------------------------------------
\section{Back to Hypothesis}

Our hypothesis argues for a \similarity based semantic hub internal organization and \association for that of other non-hub components. By modeling \similarity processing axis by \code{SIM}\slash \code{SIG} and \association by \code{ASN} features, the \code{SIM}\slash \code{SIG}-\code{ASN} voxel-performance contrast should align with the semantic hub versus non-hub spatial map, namely a bilateral (ventrolateral) aTL centered contrast. 

By indicating the presence of a content word, thus the necessity of semantic processing, the regression model improved the voxel-models located in bilateral mid temporal pole (TP), posteroinferior temporal gyrus (pITG) including left middle fusiform gyrus (mFG) and frontopolar prefrontal cortex (fpPFC). This finding is compatible with the hypothesis as both \similarity and \association regions are revealed.

The constructed \code{SIM} features most improved voxels located in bilateral middle temporal gyrus (MTG) and left superior parietal lobule and right angular cortex, \code{SIG} in posterior inferior TG, right angular gyrus (AG) and bilateral inferior frontal gyrus pars triangularis (IFGtri). \code{ASN} improvements report voxel clusters implicated in left superolateral FG, bilateral IFGtri, left medial superior frontal gyrus, left middle cingulate cortex, pSTS and other visual areas. The \code{SIM}-\code{ASN} contrast revealed preferential models for \code{SIM} in left superior frontal cortex (SFC) and anterior cingulate cortex (ACC), and before correction bilateral pSTG, left aMTG, right mMTG. The \code{SIG}-\code{ASN} contrast found \code{SIG}'s advantage in primary auditory cortex, posterior fusiform gyrus. 

Our findings for \code{SIM}/\code{SIG} modeling \similarity are not convergent on temporal organizational properties. None of anteroventral aspect of the TL is found by any of the contrast. No prior evidences suggest SFC and ACC's implication in pure \similarity processing.  

Nevertheless, pure \association areas are confirmed to be principally located in occipital area including middle and inferior aspects of occipital cortex. Bilateral BA17/18 in calcarine, cuneus, BA19, lingual gyrus are only found related with \code{ASN}.

Before examining the validity of our hypothesis, several potential confounds impact the power of this study.

\section{Precise and Informative Semantic Feature Design}

\subsubsection{Impact of \code{CWRATE}}

\code{CWRATE} indicates the necessity of semantic retrieval and processing when a stimuli is presented. Both \similarity and \association aspects are wrapped in the feature. Most of the voxel-clusters improved by \code{CWRATE} are associated with visual recognition\slash identification (posterior fusiform), visual association (BA18/19, V2,3,4,5), premotor (rolandic oper BA6) visuomotor coordination (Precuneus, Superior Occipital). But two aTL regions in MTP are also reported, which are near by the neural fiber convergence zone. 

As \code{CWRATE} is a shared feature for \code{SIM}\slash\code{SIG} and \code{ASN} models, orthonormalizing embedding feature regressors against \code{CWRATE} suppresses a large proportion of semantic-axis-specific signal in fMRI encoding, potentially weakening the contrast between \code{SIM} and \code{ASN}.

\subsubsection{Better Constructions of \code{SIM}}

\code{SIM} models have lower regression scores than \code{ASN}, this could be caused by a limited extent of \similarity processing neurons compared to \association, or the lack of quality control of the \code{SIM} embedding. The English \code{SIM} embedding is well constructed: \code{WordNet} is widely used, the resulting embedding's quality is assured by semantic evaluation tasks. Whereas for French the ontology is built upon the algorithm-generated \code{WOLF}, which makes use of multilingual resources and is composed of translation-based synsets. Additionally, the French semantic evaluation task datasets are not tested. There are potentially more appropriate construction of a valid \similarity embedding.

From a pure computational aspect of view, \textcite{bullinariaExtractingSemanticRepresentations2012} found that removing the initial PCs of singular-value-decomposed (SVD) semantic matrices improves the performance on multiple semantic tasks (such as TOEFL, Distance Comparison, Semantic Categorization and Clustering Purity, fMRI encoding/decoding tasks are not included). In our project we did not remove the initial PCs nor did \citeauthor{bullinariaExtractingSemanticRepresentations2012} provide a practical suggestion on the number of PCs to be pruned. The influence of first PCs in obtained \code{SIM} is very pronounced, they one-hot encode POS information, so that words are organized by grammatical categories in different linearly dissociable sub-spaces. As we are yet unclear on whether the human brain recruits different neural structures for words of different grammatical categories, removing the first dominant PCs of \code{SIM} might better approximate the argued \similarity axis. For example, \code{SIG}, which is not an embedding resulting from a PCA, thus have no dominant dimensions in the embedding. \code{SIG} has a greater performance compared to \code{SIM}. Purely by promoting voxel-model scores, \code{SIG} revealed more voxel clusters in contrast with \code{ASN} (despite the improved voxels are not essentially the same with \code{SIM}).

\subsubsection{Corpus and Embedding Compatibility}

For out-of-vocabulary words in the \citetitle{desaint-exuperyPetitPrince1943} (around 5\% of the vocabulary), null vectors are used to substitute (unknown) semantic values in this project. The semantic vectors however could be approximated using synonyms or associates available in embeddings to provide a more informative design matrix baseline. The selection of alternative words should be compatible to the semantic axis of the embedding in question.

\subsubsection{Design Matrix and Regression Model} 

Since GLM and Ridge regression are used, the classic problem of overfitting with a small dataset is persistent through out the project. As a tentative improve voxel-model performances, the step-wise forward feature selection is adopted. However, this scheme penalizes voxels of high-level semantic processing as low-level feature are also supplied to the regression solver, thus adding abundant dependent noises (due to orthonormalization). 

We initially considered \similarity and \association as two balanced counterparts of semantic processing, thus the contrast between models follow a non-nested design to avoid regression overfitting and computational considerations. The contrast between embedding models and non-embedding models is nested, thus all embedding-related voxel-clusters could be reported. The contrast between different embedding models rules out embedding related yet not specific regions. However, the non-nested comparison's stability and sensitivity towards weaker embeddings (in this case \similarity embeddings) are still to improve. 

More robust regression models (e.g. randomized bagging models\footnote{Features are selected randomly to produce different regression models and the prediction is the aggregation of sub-models' prediction.} and gradient boosting\footnote{Sequences of small and weak regression models are trained on the difference of the previous model's prediction and the truth value, so that the sum of the model prediction sequence minimizes the prediction error.}) exist to counter the feature selection problem inside the regression. But they are more resource-consumptive and lack explicability. Ridge regression, which has a closed-form solution, is quicker to solve and more transparent, yet less powerful. 

\section{Limits of fMRI}

\subsubsection{Ventral BOLD Signal Recording}

The adopted multi-echo fMRI sequence is adopted to better extract BOLD signals in ventral cortical areas. Traditional fMRI imaging suffers a low signal-to-noise ratio in the region due to the sinuses located near temporal poles, unable to reveal neural activations \parencite{devlinSusceptibilityInducedLossSignal2000}. The fMRI data is already capable of showing contrasts in anterior TL, which is not the case for mono-echo fMRI. The effect of \similarity and \association contrast, however, might be subtle (due to the construction of \code{CWRATE}). It could be suspected that the minute residual contrast in anteroinferior temporal lobe could not be shown by fMRI.

\subsubsection{Temporal Dynamics of Two Semantic Axes}

\textcite{lambon-ralphNeuralComputationalBases2017} states that in ventroanterior temporal lobe, domain-level semantic distinctions are available around 120 ms post stimulus onset, and around 250 ms detailed semantic information is activated. \textcite{shimotakeDirectExplorationRole2015} used local field potential evidences to show that a N300 signal is linked to ventral aTL semantic processing. 

Neither do other investigations in \similarity and \association contrast report a ventroanterior temporal contrast but a anterior temporal pole (TP), precunueus and angular (AG) contrast for \similarity, posterior fusiform (pFG) and middle/posterior STG sites for \association~\parencite{kutasBrainPotentialsReading1984, frankWordPredictabilitySemantic2017}. The reported sites are compatible with our finding, but the temporality revealed by EEG for both conditions is N400. 

In our project the anteroinferior temporal contrasts are captured by \code{CWRATE}, which could be considered as a coarse mixture of \similarity and \association. It could be suspect that the semantic hub is responsible for both principles, while \similarity precedes \association processing in the time. If the later \association activate overlaps with \similarity signals, the fMRI temporal resolution is not sufficient to capture the crucial contrast during a short time window of around 130 -- 180 ms.

\subsubsection{Beyond Lexical Semantics}

Word-meanings are essential for natural language comprehension as they serve as the foundation for phrasal and sentential understanding. \textcite{jainIncorporatingContextLanguage2018} used a deep language model to incorporate context into semantic embeddings and correlated cortical regions with the context lengths: they found voxels' preference for short context only near primary auditory (PA) cortices, left temporo-parietal junction and Broca's area. Other voxels prefer long contexts. \textcite{verdierEncodageActiviteNeuronale2018} compared the deep language model performance's in fMRI encoding with statistical word embeddings (GloVe) but the improvement was not significant.

However as we argue that GloVe itself is a mixture of syntagmatic and paradigmatic information thus the context information is partially present in the semantic vectors. \code{SIM} and \code{SIG} are argued to be free of syntagmatic information, they are thence purer lexical models. In addition to the conducted word-pair semantic proximity evaluation tasks, it is also interesting to contrast \emph{similarity}, \emph{association} and explicit context-integrated models' performance on sentence comprehension tasks.

% In a post-hoc analysis, we correlated different subject-wise model scores trained in 9 cross-validations with each participant's multi-choice comprehension questions (Table \ref{tab:behavioral}). The voxel-models' maximum and mean scores of the semantic embedding regressors are negatively correlated with question correct rate. \code{SIM} and \code{SIG}'s correlations are the most extreme (p<0.05 for both max and mean), \code{ASN}'s mean is near-significance. \code{MIX} is not significant but a negative trend is shown. 

% The findings suggest that [TODO, ANOVA] the semantic understanding modeling should not constrained by lexical semantics. To acquire comprehension of a full phrase, sentence or passage, the brain should transform and integrate each lexicon unit's semantic value. A high semantic model performance might be an indicator of dominance of mental representations for lexical semantics, and the phrasal integration of semantic values is to a limited extend. The inequivalent contribution of \similarity and \association in phrasal semantic processing is also suggested, as \similarity principle evolves on \emph{absentia} of a lexical unit's occurrence, with \association pertains a more global view on the present passage.

% Again, if lexical semantic information is present post stimulus, it could be better investigated in a tight time window , and the contrast between \similarity and \association lexical semantic values might be more clear.

\section{Statistics}

The threshold in whole-brain voxel model performance visualization (\code{r2} maps) is fixed at 0.005. This choice was arbitrary, and its utility is to filter out uninformative voxels without considering statistical significance of the regression results. As different voxels had different preferential feature dimensions across cross-validation sessions, individuals and models, the group level of significance test for \code{r2} was a complex question. Future steps of the project could F-test the \code{r2} against null distributions, or compute Monte-Carlo alternative models to test the significance. Since such tests for \code{r2} are not computed for model regression results, the cluster analysis was performed with selections of a certain proportion of best modeled voxels.

The F-test result presentation on nested-model improvement contrast is also controversial as it manipulates p-values and is relaxed to counter for individual variability.
%  The original design was to present a superposition of significance maps of each individual so that regularities could be tracked, and the geometric mean of p-values is equivalent to algorithmic mean of log p-values. The statistical robustness however is questionable. 

In this project the contrast of model performances lacks in statistical significance: the voxel-model model contrasts had p-values < 0.001 uncorrected, but none survived voxel-wise multi-comparison correction. A small effect size was foreseen since \similarity and \association contrasts are minute. However given the time constraint of the project, recruiting more subjects for additional fMRI recording was not a viable option.

\section{Cognitive Accounts on Coherence between Embeddings, Semantic Principles and Semantic Hub}

We name \similarity the internal organization of the presumed semantic hub, and argued that \similarity is principally constituted with paradigmatic axis proposed by \citeauthor{jakobsonFundamentalsLanguage1963}. Additionally with pathological evidences, multiple properties of \similarity axis are defined: cross-modality and conceptual hierarchy, conformable with properties of WordNet-alike ontologies. Semantic evaluation tasks based on word-pair proximity evaluation suggest the validity of \code{SIM}/\code{SIG} model against \emph{similarity}, especially for English embeddings where \code{WordNet} and evaluation benchmarks are well founded.

Yet no effective evidence confirms the bridging of \similarity and semantic hub. 

\subsubsection{Success in \emph{Association} Modeling?}

\emph{Association} is proposed as an umbrella term for all non-\similarity information. Thus \code{ASN} embeddings are built as the residual of subtraction of \similarity embeddings from a mixed embedding. However, since we used GloVe and DepGloVe as our mixed embedding, the corpora used to build these two embeddings are purely textual, thus no explicit perceptual data are provided. An embedding space, composed majorly by syntagmatic information, found its better encoding voxel-model in multiple primary visual areas alongside with visual association areas (bilateral BA17, 18, 37) when contrasted with \similarity (No visual area is reported by contrasting \code{ASN} with non-embedding features). This finding is convergent with our hypothesis on \association constructions: modality-dependent, association with episodic memories. Thus the modality-independent aspect of our \similarity embedding models, which is presumed to extract the rest of information, can be partially confirmed.

\subsubsection{Pathway to Semantic Hub: Accumulative or Differential?}

In our hypothesis, we presumed that the semantic hub holds a global view of all representational or operational semantic information, including specific, basic-level and domain-level concepts (Section \ref{subsection:hypsemantichub}). Correspondingly, a holistic \similarity embedding is constructed, containing all semantic entities. Yet, such construction underestimates the participation of non-hub structures (which are domain-specific or feature-specific). 

If more coarse semantic representation is available in the semantic hub earlier then the detailed information, does the semantic hub keep the coarse copy, or the domain-\slash feature-specific spokes jointly participate in the representation? \textcite{clarkePerceptionConceptionHow2013}'s MEG data suggests that both general/conceptual and modality-specific can be linked in left ventral temporal cortices after the full semantic activation, yet the spokes during the time course is also correlated with both general and specific information. 

If the spokes and the hub exchange information and the detailed representation is eventually available in both regions, the \similarity principle proposed by this project corresponds not only to the strict semantic hub, but also to neural structures connecting to the hub loci. One possible investigation to contrast the hub is to look into the spatial distribution of effective feature-dimension of voxel-models of \code{SIM} (similar to \textcite{huthContinuousSemanticSpace2012} yet the objective is to find a hub which is linked with most of the features). 
