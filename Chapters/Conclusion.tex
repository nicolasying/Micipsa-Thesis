\chapter{Conclusion} % Main chapter title

\label{chap:conclusion} 

In this project, we propose two types of semantic embedding spaces encoding respectively semantic \similarity and \association. We selected a set of \similarity semantic relationships and converted semantic ontologies to similarity embeddings. We proposed the usage of general linear model to dissociate \similarity information from \association, which are mixed in classic statistical distributed embeddings. Each constructed space contains pure semantic information of one semantic axis, confirmed by semantic evaluation tasks specifically constructed for each axis and example examinations. The collected evidences suggest that the \similarity embedding construction and GLM dissociation methods are valid.

With built embeddings, we try to replicate the anterotemporal localization of semantic hub by contrasting \similarity voxel-models with \association ones. The voxel-models are trained with semantic embeddings, combined with basic features to encode fMRI BOLD signals. While \similarity embeddings find mostly middle temporal, superior parietal and angular improvements when contrasted with basic features, and superior/middle frontal, anterior cingulate, superior/middle temporal with \association features, \association models found occipital, frontal, middle temporal, inferior triangular frontal contrasted with basic features, and inferior temporal, occipital and parahippocampal improvements contrasted with \emph{similarity}. While the results are expected for \association since associative cortical areas are correlated, the argued aTL loci is not supported by the data.

We argue that the contrast between \similarity and \association might be improved with a better construction of \similarity embeddings and encoding design matrices on computational linguistic accounts. Other imaging methods other than fMRI might better show a temporal aspect of the two-axis contrast. Imaging methods with higher temporal resolutions might also better investigate lexical semantic processing. Finally, we reviewed our hypothetical structure of \similarity of the semantic hub, proposing alternative semantic hub localization methodology based on effective semantic feature regression analysis.