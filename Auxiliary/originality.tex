\begin{declarationauthorship}
    \addchaptertocentry{Declaration of Originality} % Add the declaration to the table of contents
    \noindent This master's project consists of an original research on the dissociation of multiple organization principles of human semantic processing. Namely, we relate various language-related cortical areas with different semantic functions using an fMRI encoding experiments. 
    
    This study differs from existing works in computational linguistics in the following points.
    \begin{itemize}
    \item It attempts to build non-generic semantic word embeddings, targeting specific semantic principles parallel to the paradigmatic and syntagmatic axes. Thus it requires fine tuning certain parameters when configuring existing embedding generation algorithms.
    \item Mathematic operations are applied on different types of semantic spaces to extract new embedding spaces. These manipulations show interactions between different embeddings.
    \item New baseline benchmark datasets are made available for French word-pair semantic proximity evaluation.
    \end{itemize}
    
    This study further investigates current theories and discoveries on a hypothetical function locus of semantic processing and multiple cortices' contribution in verbal comprehension with fMRI data.
    \begin{itemize}
    \item Existing works in semantic fMRI encoding either use non-ecological stimuli to reveal semantic condition contrasts, or ecological stimuli to reveal general semantic processing without targeting different (hypothetical) semantic aspects.
    \item A possible construction of \similarity and \association semantic memories are proposed and tested, which is based on evidences and theories on paradigmatic axis/semantic hub/convergence zone and syntagmatic axis\slash associational activation.
    \item The anterior temporal lobe localization hypothesis for a central semantic processing component is tested with ecological fMRI encoding.
    \item This is the first project analyzing French fMRI data collected in the project ``Neural Computational Models of Natural Language'' (PI: John Hale and Christophe Pallier). Precedent projects were performed with English data.
    \end{itemize}
    
    The fMRI encoding pipeline also differs from most other works in the following aspects.
    \begin{itemize}
    \item When training voxel-models for BOLD prediction, a large grid-search for best regression parameters is carried out so that each voxel is modeled by the most appropriate functional features. 
    \item Multiple condition contrasting methods are employed including nested-model improvement testing and non-nested model performance comparison.
    \end{itemize}

    \end{declarationauthorship}
    
    \clearpage